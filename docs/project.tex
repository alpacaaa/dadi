\documentclass[a4paper]{article}
\usepackage{enumitem}

\author{Stefano Rodighiero}
\title{Dadi}

\begin{document}
\maketitle
\tableofcontents

\section{Scenario}

In un gioco di ruolo i giocatori costruiscono colletivamente una
storia. Nella maggioranza dei casi il meccanismo di gioco prevede la
presenza di un Narratore (chiamato alternativamente Master e talvolta
Arbitro) che descrive la situazione agli altri giocatori, interpreta i
personaggi non giocanti (antagonisti e in generale tutti i personaggi
non controllati dai giocatori) e applica il regolamento per risolvere
i conflitti che avvengono nel gioco. Gli altri Giocatori interpretano
ciascuno un Personaggio, reagendo alle situazioni proposte dal
Narratore e facendo proseguire la storia con le loro azioni.

Esistono molteplici sistemi (regolamenti) che guidano le decisioni dei
giocatori e descrivono come risolvere i conflitti che capitano durante
il gioco. \`E comune usare un elemento aleatorio (ad esempio il lancio
di un dado) per simulare la parziale imprevedibilit\`a dell'esito
delle azioni.

Nel gioco tradizionale i giocatori si riuniscono nello stesso luogo e
la risoluzione dei conflitti \`e demandata a qualche genere di
processo fisico verificabile da tutti: lancio di un dado, estrazione
di una carta o cose simili. Per il nostro esercizio, invece,
immaginiamo un ambiente di gioco distribuito geograficamente: i
giocatori ricorrono alla rete per giocare insieme, e hanno bisogno di
un sistema centralizzato e imparziale per regolare i conflitti. Lo
scopo dell'esercizio \`e realizzare tale sistema.

Tale sistema prevede:

\begin{itemize}
\item Creazione di check da parte del Narratore. Un check \`e
  definito da una probabilit\`a di riuscita e da una descrizione.
\item Consultazione dell'elenco dei check, aperti o gi\`a risolti.
\item Provare a risolvere un check. Tale operazione modifica lo stato
  del check. Successive consultazioni di questo check mostrano la
  probabilit\`a di riuscita definita, e l'esito della prova avvenuta.
\end{itemize}

Per rendere l'idea, ecco la trascrizione di una ipotetica sessione di
gioco che utilizza questo sistema.

\begin{enumerate}[leftmargin=3.5cm]
\item [\textbf{Dario:}] Siete in una radura. A Nord c'\`e una parete di
  roccia. Tra i rampicanti che la ricoprono riconoscete chiaramente il
  contorno di una porta di ferro.
  
  \textit{Dario \`e il narratore.}
  
\item [\textbf{Stefano:}] Il mio personaggio prova ad aprire la porta spingendola.

  \textit{Stefano \`e un giocatore.}

\item [\textbf{Dario:}] Va bene. Il tuo personaggio per\`o non \`e
  molto forte, e la porta sembra molto pesante, sar\`a difficile.

  \textit{Dario accede al servizio web per definire un
    check. Stabilisce che la probabilit\`a di riuscita \`e pari a
    20\%. Ottiene in cambio la URL del check appena creato.}

\item [\textbf{Dario:}] Stefano, ecco la URL del check che devi passare.

  \textit{Stefano punta il suo browser sulla URL ricevuta. Trova una
    descrizione dell'azione che vuole provare, e la probabilit\`a di
    riuscita stabilita da Dario. C\`e un pulsante per tentare il
    check. Purtroppo l'esito \`e negativo.}

\item [\textbf{Stefano:}] Purtroppo non ci sono riuscito!

  \textit{Dario ricarica la URL che ha passato a Stefano. Poich\`e il
    check \`e gi\`a stato risolto, vede solo l'esito della prova.}

\end{enumerate}

\section{Il progetto}

Il progetto proposto \`e composto da una API REST e da una interfaccia
Web per creare, consultare e risolvere check. Propongo di usare il
framework Servant per costruire il backend.

Serve un layer di persistenza per memorizzare i dati che definiscono i
check.

Non ho le idee chiare a proposito della realizzazione dell'interfaccia
web, ma probabilmente basta un sistema di template che vada bene
accoppiato a Servant.

\section{Specifiche}

\subsection{Entit\`a e azioni dell'interfaccia REST}

\begin{itemize}
\item \url{/check/new} \\ Una chiamata GET restituisce un form dove
  inserire i parametri che definiscono il check.

  Una chiamata POST crea un nuovo check e restituisce un
  identificatore per il check appena creato, o un errore se i
  parametri forniti sono incompleti o malformati. Il payload della
  chiamata POST \`e un documento JSON che contiene i parametri di
  creazione del check: la descrizione \`e una stringa di testo
  arbitraria, la probabilit\`a di riuscita \`e un valore numerico.

\begin{verbatim}
{
  description: "Provi a spingere la porta di ferro."
  prob: 20
}  
\end{verbatim}

\item \url{/check/:id} \\ Una chiamata GET restituisce una pagina con
  la descrizione completa del check e un pulsante per provare a
  passarlo.

  Una chiamata PUT aggiorna lo stato del check, generando un valore
  random e confrontandolo con la probabilit\`a di riuscita
  dell'azione. Una chiamata PUT su un check gi\`a risolto non ha alcun
  effetto.
\item \url{/checks} \\ Restituisce la lista di tutti i check, ordinati
  per stato e per data di creazione descrescente.
\end{itemize}

\end{document}
